\documentclass[9pt]{article}
\usepackage{CJK}
\usepackage{fullpage}
\usepackage{amsmath}
\usepackage{amssymb}
\usepackage[usenames]{color}
\usepackage[colorlinks,linkcolor=black]{hyperref}
<<<<<<< HEAD

% Set margins
\leftmargin=0.25in
\oddsidemargin=0.25in
\textwidth=6.0in
\topmargin=0in
\textheight=9.25in

=======

% Set margins
\leftmargin=0.25in
\oddsidemargin=0.25in
\textwidth=6.0in
\topmargin=0in
\textheight=9.25in

>>>>>>> 6a3a2a3a42d536ae20103189d88ff56d732f039e
\raggedright

%\pagenumbering{arabic}
\thispagestyle{empty}

\def\bull{\vrule height 0.8ex width .7ex depth -.1ex }
% DEFINITIONS FOR RESUME

\newenvironment{changemargin}[2]{%
  \begin{list}{}{%
    \setlength{\topsep}{0pt}%
    \setlength{\leftmargin}{#1}%
    \setlength{\rightmargin}{#2}%
    \setlength{\listparindent}{\parindent}%
    \setlength{\itemindent}{\parindent}%
    \setlength{\parsep}{\parskip}%
  }%
  \item[]}{\end{list}
}

\newcommand{\lineover}{
    \begin{changemargin}{-0.05in}{-0.05in}
        \vspace*{-8pt}
        \hrulefill \\
        \vspace*{-2pt}
    \end{changemargin}
}

\newcommand{\header}[1]{
    \begin{changemargin}{-0.5in}{-0.5in}
        \scshape{#1}\\
    \lineover
    \end{changemargin}
}

\newcommand{\contact}[5]{
<<<<<<< HEAD
    \begin{changemargin}{-0.5in}{-0.5in}
        \begin{center}
            {\Large \scshape {#1}}\\ \smallskip
            {#2}\\ \smallskip 
            {#3}\\ \smallskip
            {#4}\\ \smallskip
            {#5}\smallskip
        \end{center}
    \end{changemargin}
=======
	\begin{changemargin}{-0.5in}{-0.5in}
		\begin{center}
			{\Large \scshape {#1}}\\ \smallskip
			{#2}\\ \smallskip 
			{#3}\\ \smallskip
			{#4}\\ \smallskip
			{#5}\smallskip
		\end{center}
	\end{changemargin}
>>>>>>> 6a3a2a3a42d536ae20103189d88ff56d732f039e
}

\newenvironment{body} {
    \vspace*{-16pt}
    \begin{changemargin}{-0.25in}{-0.5in}
  } 
    {\end{changemargin}
}   

\newcommand{\school}[4]{
    \textbf{#1} \hfill \emph{#2\\}
    #3\\ 
    #4\\
}

% END RESUME DEFINITIONS

% Add your personal contents from here!!! 

% Several instructions: 

% \textbf{} bold
% \emph{} or \textit{} italic type
% \underline{} underline
% %--the compiler ignores the text behind '%'
% Set font size (from small to large)
% \tiny
% \scriptsize
% \footnotesize
% \small
% \normalsize
% \large
% \Large
% \LARGE
% \huge
% 
% More information, just google that with keyword "Latex your question"


\begin{document}
\begin{CJK*}{UTF8}{gbsn}

%%%%%%%%%%%%%%%%%%%%%%%%%%%%%%%%%%%%%%%%%%%%%%%%%%%%%%%%%%%%%%%%%%%%%%%%%%%%%%%%
% Name
<<<<<<< HEAD
\contact{耿天毅}{(+86) 177-0130-8306}{gty12@mails.tsinghua.edu.cn}{\href{http://tianyi.world}{个人网站:http://tianyi.world}}
=======
\contact{耿天毅}{北京市海淀区清华大学紫荆学生公寓1号楼129B,邮政编码:100084}{(+86) 188-1060-1766}{gty12@mails.tsinghua.edu.cn}{\href{http://TarnumG95.github.io}{网站:http://TarnumG95.github.io}}
>>>>>>> 6a3a2a3a42d536ae20103189d88ff56d732f039e

%%%%%%%%%%%%%%%%%%%%%%%%%%%%%%%%%%%%%%%%%%%%%%%%%%%%%%%%%%%%%%%%%%%%%%%%%%%%%%%%
% Education
\header{教育}

\begin{body}
<<<<<<< HEAD
    \vspace{14pt}

% -----
    清华大学\ 电子工程系 \hfill 中国北京 \\
=======
	\vspace{14pt}

% -----
	清华大学 电子工程系 \hfill 中国北京 \\
>>>>>>> 6a3a2a3a42d536ae20103189d88ff56d732f039e
工学学士 \hfill 2012.8 - 今 \\
学习成绩:9*.**/100.00\ \ \ 排名:*/241\\
兴趣:信号处理,数据挖掘 \\
\vspace{6pt}

% -----
<<<<<<< HEAD
    清华大学\ 经济管理学院 \hfill 中国北京 \\
管理学学士(第二学士学位)\hfill 2013.8 - 今 \\
\vspace{6pt}

% -----
    威斯康星大学麦迪逊分校\ 电子与计算机工程系\hfill 美国威斯康星州 \\
优秀本科生国际交流项目 \hfill 2014.8 - 2014.12 \\
=======
	清华大学 经济管理学院 \hfill 中国北京 \\
管理学学士 \hfill 2013.8 - 今 \\
\vspace{6pt}

% -----
	威斯康星大学麦迪逊分校 工学院 \hfill 美国威斯康星州 \\
交换生,优秀本科生国际交流项目 \hfill 2014.8 - 2014.12 \\
>>>>>>> 6a3a2a3a42d536ae20103189d88ff56d732f039e
学习成绩:3.92/4.00

\end{body}

\smallskip
\smallskip
\smallskip

%%%%%%%%%%%%%%%%%%%%%%%%%%%%%%%%%%%%%%%%%%%%%%%%%%%%%%%%%%%%%%%%%%%%%%%%%%%%%%%%
% Skills
\header{技能}

\begin{body}
<<<<<<< HEAD
    \vspace{14pt}
    % -----
    计算机技能:
    \begin{itemize}
    \itemsep 0pt
    \item 熟练:C/C++、Python、MATLAB;
    \item 熟悉:Hadoop、R、Java、Linux、Mac OS X;
    \item 初学:HTML、JavaScript;
    \end{itemize}

    \smallskip
% -----
    数学:熟悉微积分,线性代数,概率论与随机过程,离散数学,算法;\\
    \smallskip
% -----
    电子工程:熟悉信号处理,图像处理;\\
    \smallskip
    

    % -----
    语言:熟练掌握英语(雅思7.0分)。\\
=======
	\vspace{14pt}
% -----
	数学:熟悉微积分,线性代数,概率论与随机过程,离散数学,算法;\\
	\smallskip
% -----
	电子工程:熟悉信号处理,图像处理;\\
	\smallskip
	% -----
	计算机:熟悉C/C++、Python、R、MATLAB、Mathemetica、Linux、Mac OS X,初学HTML、JavaScript;\\
	\smallskip

	% -----
	语言:熟练掌握英语(雅思7.0分)。\\
>>>>>>> 6a3a2a3a42d536ae20103189d88ff56d732f039e
\end{body}
\smallskip
\smallskip
\smallskip

%%%%%%%%%%%%%%%%%%%%%%%%%%%%%%%%%%%%%%%%%%%%%%%%%%%%%%%%%%%%%%%%%%%%%%%%%%%%%%%%
% Projects
\header{项目经历}

\begin{body}
<<<<<<< HEAD
    \vspace{14pt}
% -----
    定位图片中水印形式的直线 \hfill 2014.9 - 2014.11
    \begin{itemize}
    \itemsep 0pt
    \item 设计了准确而高效的检测直线水印的算法;
    \item 应用了含高斯滤波器、统计学习在内的多种技术处理图片;
    \item 演示:\href{http://TarnumG95.github.io/chainlinedemo.html}{http://TarnumG95.github.io/chainlinedemo.html}。
    \end{itemize}
    \smallskip
% -----
    挖掘新浪微博数据 \hfill 2015.2 - 2015.4
    \begin{itemize}
    \itemsep 0pt
    \item 用Python处理TB级别微博数据;
    \item 发掘了用户行为模式随时间的变化,例如近年来用户更倾向用移动端发微博。
    \end{itemize}
    \smallskip
    % -----
网络大数据挖掘与数据驱动的未来网络关键技术研究 \hfill 2015.3 - 今
    \begin{itemize}
    \itemsep 0pt
    \item 用Hadoop平台分析社交媒体、运营商等记录网络与用户行为的大规模真实数据;
    \item 通过数据挖掘,认识用户规律,并设计大数据驱动的下一代移动通信网络。
    \end{itemize}
    \smallskip
=======
	\vspace{14pt}
% -----
	定位图片中水印形式的直线 \hfill 2014.9 - 2014.11
	\begin{itemize}
	\itemsep 0pt
	\item 设计了准确而高效的检测直线水印的算法;
	\item 应用了含高斯滤波器、统计学习在内的多种技术处理图片;
	\item 课程班级竞赛中取得第二名;
	\item 演示:\href{http://TarnumG95.github.io/chainlinedemo.html}{http://TarnumG95.github.io/chainlinedemo.html}。
	\end{itemize}
	\smallskip
% -----
	挖掘新浪微博(中国版Tweeter)数据 \hfill 2015.2 - 今
	\begin{itemize}
	\itemsep 0pt
	\item 用Python处理TB级别微博数据;
	\item  试图发掘用户行为模式随时间的变化,例如近年来用户更倾向用移动端发微博。
	\end{itemize}
	\smallskip
>>>>>>> 6a3a2a3a42d536ae20103189d88ff56d732f039e

\end{body}
\smallskip
\smallskip

%%%%%%%%%%%%%%%%%%%%%%%%%%%%%%%%%%%%%%%%%%%%%%%%%%%%%%%%%%%%%%%%%%%%%%%%%%%%%%%%
% Awards and Honors
\header{奖励}

\begin{body}
<<<<<<< HEAD
    \vspace{14pt}
    % -----
    2015年数学建模竞赛全球二等奖 \hfill{} 2015.4\\
    \smallskip
    % -----
    国家奖学金 \hfill{} 2014.10\\
    \smallskip
    % -----
    清华大学一二.九奖学金(综合素质前2\%) \hfill{} 2013.10\\
    \smallskip
    % -----
    清华大学新生一等奖学金 \hfill{} 2012.10\\
    \smallskip
    % -----
    2012年吉林省高考理科类第一名(1/100,000) \hfill{} 2012.6
=======
	\vspace{14pt}
	% -----
	国家奖学金 \hfill{} 2014.10\\
	\smallskip
	% -----
	清华大学一二.九奖学金 \hfill{} 2013.10\\
	\smallskip
	% -----
	第30届全国大学生物理竞赛(非物理A类)二等奖 \hfill{} 2013.12\\
	
	\smallskip
	% -----
	清华大学新生一等奖学金 \hfill{} 2012.10\\
	\smallskip
	% -----
	2012年吉林省高考理科类第一名(1/100,000) \hfill{} 2012.6
>>>>>>> 6a3a2a3a42d536ae20103189d88ff56d732f039e
\end{body}

\smallskip
\smallskip

%%%%%%%%%%%%%%%%%%%%%%%%%%%%%%%%%%%%%%%%%%%%%%%%%%%%%%%%%%%%%%%%%%%%%%%%%%%%%%%%
% Major Courses
% \header{Main Courses}

% \begin{body}
<<<<<<< HEAD
%   \vspace{14pt}

%   \begin{itemize} \itemsep -0pt
%       \item Mining Massive Datasets (Stanford CS246, Coursera)
%       \item Computer Program Design -- 98, 97 (2012 Fall \& 2013 Spring)
%       \item Data Structure, Numerical Analysis and Algorithms -- 89 (2013 Fall)
%       \item Probability and Stochastic Processes(1) -- 100 (2014 Spring)
%       \item Probability and Stochastic Processes(2) -- A (UW-Madison ECE 730, 2014 Fall)
%   \end{itemize}
=======
% 	\vspace{14pt}

% 	\begin{itemize} \itemsep -0pt
% 		\item Mining Massive Datasets (Stanford CS246, Coursera)
% 		\item Computer Program Design -- 98, 97 (2012 Fall \& 2013 Spring)
% 		\item Data Structure, Numerical Analysis and Algorithms -- 89 (2013 Fall)
% 		\item Probability and Stochastic Processes(1) -- 100 (2014 Spring)
% 		\item Probability and Stochastic Processes(2) -- A (UW-Madison ECE 730, 2014 Fall)
% 	\end{itemize}
>>>>>>> 6a3a2a3a42d536ae20103189d88ff56d732f039e

% \end{body}

% \smallskip
% \smallskip

%%%%%%%%%%%%%%%%%%%%%%%%%%%%%%%%%%%%%%%%%%%%%%%%%%%%%%%%%%%%%%%%%%%%%%%%%%%%%%%%
% Experience
\header{其他}

\begin{body}
<<<<<<< HEAD
    \vspace{14pt}
    % -----
    清华大学无25班班长,学习委员 \hfill 2012.10 - 2014.9\\
    \smallskip
    清华大学本科生课程咨询委员会 \hfill 2014.12 - 今\\
    \smallskip
    % -----
    已修课程列表: \href{http://tarnumg95.github.io/course.html}{http://TarnumG95.github.io/course.html}\\
=======
	\vspace{14pt}
	% -----
	清华大学无25班班长,学习委员 \hfill 2012.10 - 2014.9\\
	\smallskip
	% -----
	已修课程列表: \href{tarnumg95.github.io/course.html}{http://TarnumG95.github.io/course.html}\\
>>>>>>> 6a3a2a3a42d536ae20103189d88ff56d732f039e


\end{body}

\smallskip
\smallskip

\end{CJK*}
\end{document}